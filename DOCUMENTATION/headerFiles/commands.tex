% Diese Datei umfasst selbst definierte oder geänderte Befehle
    % List of commands
    %   Legende
    %   L = Label
    %   K = Kapitel
    %   Sek = Sektion
    %   S = Seite
    %   B = Bezeichnung
    %
    %   Gruppe  Bezeichnung Ergebnis
    %   Referenzen
    %       Kapitel
    %           \Uref{L}                | (siehe Kapitel 3.4)
    %           \Chref{L}               | (siehe Kapitel 3.4 Name)
    %           \ChInref{L}             | Kapitel 3.4 Name
    %       Anforderungen
    %           \RequirementInRef{L}{B} | Anforderung FJ-F-<NUMMER> Name
    %           \RequirementRef{L}{B}   | (siehe Anforderung FJ-F-<NUMMER> Name)
    %       Bilder 
    %           \PicRef{L}              | Abb. <Nummer> <Kursiv Beschreibung>
    %           \PicInRef{L}            | Abb. <Nummer>
    %       Zitate                      
    %           \citeCh{L}{K}           | (..., siehe Kap. ...)
    %           \citeChSec{L}{K}{Sek}   | (..., siehe Kap. ... Abs. ....)
    %           \citePage{L}{S}         | (..., vgl. S. ...)
    %
    %
    %   Kapitel
    %      Erstellt einen Eintrag im Inhaltsverzeichnis und vor dem Kapitel
    %       \DefineNewChapter{Name} | Geeignet für Einleitung, Sperrvermerk, Anhang, Abkürzungsverzeichnis
    %       \DefineNewListOf[label]{Name}  | Geeignet für Abbildungsverzeichnis, Tabellenverzeichnis, ...
    %

%   Listen
%       \DefineFktAnforderung{Headline]{Beschreibung}{Labelsuffix}
%       \DefineNichtFktAnforderung{Headline}{Beschreibung}{Labelsuffix}

%\usepackage{tocloft}% http://ctan.org/pkg/tocloft

\makeatletter % needed because of the @
    \newcommand\storelabel[1]{\def\@currentlabel{#1}}
\makeatother



\newcounter{DefineFktAnforderung}
\setcounter{DefineFktAnforderung}{0}

% \AnforderungDesign{Präfix}{Headline}{Text}{Label}{Counter}
\newcommand{\AnforderungDesign}[5]{
    \noindent\textbf{#1-#5 ~#2}\storelabel{#1-#5 ~#2}\label{Anf:#4} \\
    \noindent\rule[5mm]{\textwidth}{1pt} \\
    \vspace{-15mm} 
        
    \noindent {#3}\nolinebreak
    \vspace{5mm}
}

%\newlistof{DefineFktAnforderung}{DefineFktAnforderungDef}{\small Liste der funktionalen Anforderungen}
% \DefineFktAnforderung{Headline}{Describtion}{Labelsuffix}
\newcommand{\DefineFktAnforderung}[3]{
    \refstepcounter{DefineFktAnforderung}

    \AnforderungDesign{F}{#1}{#2}{#3}{\theDefineFktAnforderung} \\
    
    %\addcontentsline{DefineFktAnforderungDef}{figure}
    %{\protect\numberline{F-\theDefineFktAnforderung}#1}\par%
}


\newcounter{DefineNotFktAnforderung}
\setcounter{DefineNotFktAnforderung}{0}

%\newlistof{DefineNotFktAnforderung}{DefineNotFktAnforderungDef}{\small Liste der nicht-funktionalen Anforderungen}
% \DefineFktAnforderung{Headline}{Describtion}{Labelsuffix}
\newcommand{\DefineNichtFktAnforderung}[3]{
    \refstepcounter{DefineNotFktAnforderung}

    \AnforderungDesign{NF}{#1}{#2}{#3}{\theDefineNotFktAnforderung} \\
    
    %\addcontentsline{DefineNotFktAnforderungDef}{figure}
    %{\protect\numberline{F-\theDefineNotFktAnforderung}#1}\par%
}


%Referenzen 
    % Universal Referenz mit angepasstem Text (siehe Kapitel 3.4)
        \newcommand{\Uref}[1]{(siehe \cref{#1})}
    % Kapitel Referenz mit angepasstem Text (siehe Kapitel 3.4 Name)
        \newcommand{\ChRef}[1]{(siehe \cref{#1} \textit{\nameref{#1}})}
    % Kapitel Referenz Inline mit angepasstem Text Kapitel 3.4 Name
        \newcommand{\ChInRef}[1]{\cref{#1} \textit{\nameref{#1}}}

        
    % Bild Referenz mit angepasstem Text Abb. <Nummer> <Kursiv Beschreibung>
        \newcommand{\PicRef}[1]{\cref{#1} \textit{\nameref{#1}}}
    % Bild Referenz mit angepasstem Text Abb. <Nummer> 
        \newcommand{\PicInRef}[1]{\cref{#1}}

        
    % Anforderungs Referenz mit angepasstem Text (siehe Anforderung F-<NUMMER> Name)
        \newcommand{\RequirementRef}[1]{(siehe Anforderung \textit{\ref{Anf:#1}})}
    % Anforderungs Referenz mit angepasstem Text Anforderung F-<NUMMER> Name
        \newcommand{\RequirementInRef}[1]{Anforderung \textit{\ref{Anf:#1}}}

% Zitierung
    % Kapitel zietieren (..., siehe Kap. ...)
        \newcommand{\citeCh}[2]{\parencite[siehe Kap. #2]{#1}}
    % Kapitel mit Sektion zietieren (..., siehe Kap. ... Abs. ....)
        \newcommand{\citeChSec}[3]{\parencite[siehe Kap. #2 Abs. #3]{#1}}
    % Mit Seiten zietieren (..., vgl. S. ...)
        \newcommand{\citePage}[2]{\parencite[vgl. S. #2]{#1}}





% Anpassung der Chapter im Inhaltsverzeichnis 
    % \DefineNewChapter{Name}
    % (geeignet für Einleitung, Sperrvermerk, Anhang, ...)
    \newcommand{\DefineNewChapter}[1]{
        \phantomsection 
        \addtocontents{toc}{\vspace{-1.5ex}}
        \addcontentsline{toc}{chapter}{#1}
        \addchap*{#1}
    }

    % (geeignet für Abbildungsverzeichnis, Tabellenverzeichnis, ...)
    % \DefineNewListOf{label}[Name]
    \newcommand{\DefineNewListOf}[2][]{
        \phantomsection 
        \ifthenelse{\isempty{#1}}{}{\label{#1}}
        \addtocontents{toc}{\vspace{-1.5ex}}
        \addcontentsline{toc}{chapter}{#2}
    }



