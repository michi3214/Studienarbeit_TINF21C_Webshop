\usepackage[utf8]{inputenc}
\usepackage[T1]{fontenc}
\usepackage[ngerman]{babel}

% Ermöglicht das Erstellen von comment Blocks
    \usepackage{comment} % Kommentarblöcke mit \begin{comment}

% Eine Tabelle kann die größe einer Seite überschreiten
    \usepackage{longtable} % Used for a table longer one page

    
%Import von Bildern
    \usepackage{graphicx}
    % Position der Bilder
        \graphicspath{ {./images/} }
    % Ausrichten von Bildern 
        \usepackage[export]{adjustbox}


% unterschiedliche Schriftarten
    %\usepackage{lmodern}
    %\usepackage{palatino} 
    %\usepackage{tgheros}
    %\usepackage{goudysans}
    %\usepackage{libertine}
    %\usepackage[sfdefault]{AlegreyaSans}
        %\renewcommand*\oldstylenums[1]{{\AlegreyaSansOsF #1}} % gehört zu AlegreyaSans
    %\usepackage[sfdefault]{biolinum}
    %\usepackage{courier}


% Striche bei der Fuß-, sowie Kopfzeile
    \usepackage[footsepline,plainfootsepline, headsepline, plainheadsepline]{scrlayer-scrpage} 
    \pagestyle{scrheadings}
    \clearpairofpagestyles
    \KOMAoptions{
        headheight = 1cm,
        footheight = 1cm,
        singlespacing  =true
    }
    
% Größen der Überschriften
    \setkomafont{chapter}{\Large}
    \setkomafont{section}{\large}
    \setkomafont{subsection}{\normalsize}
    \setkomafont{subsubsection}{\small}

% viele Mathe-Symbole
    \usepackage{amssymb}
    % Erweiterungen für amsmath
        \usepackage{mathtools} 


% Links
    \usepackage{hyperref}
    
        % Durch diese Einstellung werden Links hervorgehoben durch die gegebenen Farben
        %\hypersetup{
        %   colorlinks,
        %   linkcolor={red!50!black}, % color of internal links
        %   citecolor={blue!50!black},% color of links to bibliography
        %   urlcolor={blue!80!black}  % color of external links
        %}
    
    % Referenz zu Bildern mit Ansicht auf Bilder
    \usepackage[figure]{hypcap}
    % Unter einer Abbildung steht nun Abb.
    \addto\captionsngerman{%
      \renewcommand{\figurename}{Abb.}%
    }

    % Referenzen in der Datei
        \usepackage[nameinlink]{cleveref}
                
    	% nicht nötig da ngerman in document class definiert
            \crefname{lstlisting}{Listing}{Listings}
            %\crefname{chapter}{Kapitel}{Kapitel}
            %\crefname{section}{Abschnitt}{Abschnitte}
            %\crefname{subsection}{Unterabschnitt}{Unterabschnitte}
            %\crefname{figure}{Abbildung}{Abbildungen}
        

% Ein Durchlauf bei der Erstellung von Verzeichnissen
\usepackage{bookmark}
\usepackage{makeidx}

% Einstellung des allgemeinen Zeilenabstands und die möglichkeit zur Anpassnung mit spacing
\usepackage[onehalfspacing]{setspace}

% Anführungszeichen
\usepackage[babel,german=quotes]{csquotes}

%Unterstrich für die Unterschrift / Tabellen erstellung
\usepackage{tabularx}
% Tabellenformatierung
    \usepackage{multirow}

% Literaturverweise
    \usepackage[style=authoryear, % Style Definition
                backend=biber,    % Backend Definition
                minnames=3,       % Minimale Anzahl an Namen
                maxbibnames=3,    % Maximale Anzahl an Namen
                dashed=false,     % 
                sorting=nty       % Sortierung des Verzeichnisses nach name, title, year.
                ]{biblatex}
    % In dieser Datei erfolgt die Definition der Quellen 
        % In dieser Dateien sind die Dateien mit den Literturinhalten angeben
        \addbibresource{verzeichnisse/Literatur/books.bib} 
        %\addbibresource{verzeichnisse/Literatur/internals.bib} 
        %\addbibresource{verzeichnisse/Literatur/website.bib} 
        %\addbibresource{verzeichnisse/Literatur/misc.bib} 
        %\addbibresource{verzeichnisse/Literatur/manual.bib} 
        \addbibresource{verzeichnisse/Literatur/RFC.bib} 
    % Komma im Text bei cite zwischen den Daten (Standard .)
        \DeclareDelimFormat{nameyeardelim}{\addcomma\space}
    % Komma im Literaturverzeichnis zwischen den Daten (Standard .)
        \renewcommand*{\newunitpunct}{\addcomma\space} 
    % Zeichen hinter jeder Literatur im Verzeichnis (Standard .)
        \renewcommand*{\finentrypunct }{} 
    % Bezeichnung bei mehreren Autoren
        \DefineBibliographyStrings{ngerman}{andothers = {{u. a.},}}
    %Abstand im Literaturverzeichnis
        % Allgemein
            \setlength{\bibitemsep}{0.2\baselineskip}
        % Gruppiert nach Namen
            \setlength{\bibnamesep}{0.6\baselineskip}
        % Abstand zwischen anderen Anfangsbuchstaben
            %\setlength{\bibinitsep}{0.6\baselineskip}



%Tabellen Einstellungen
    % Fußnote in einer Tabelle
    \usepackage{threeparttable}
    %Tabellen/Bilder Überschrift/Unterschrift
    \usepackage{caption}
    %Anpassen der Float-Umgebungen 
    \usepackage{float} % float kann mithilfe von capition entfernt werden (Nutze beides)
    %Umbennen der Bezeichner
    \addto\captionsngerman{
        \renewcommand{\tablename}{\small{Tab.}}
    }
    

%Farb Packet (Wird hauptsächlich beim Code highlighting genutzt) 
    \usepackage[dvipsnames]{xcolor}

% Für Codehighlighting
    \usepackage{listings}
    \definecolor{backgoundColor}{rgb}{.9,.9,.9} % lightgray 

    % Programmiersprachen Definitionen
    % used by listings -> siehe header.text

\definecolor{yellow}{rgb}{0.93, 0.57, 0.13}

\lstdefinelanguage{JavaScript}{
  %keywordstyle={[2]\color{White}},
  morekeywords={typeof, new, true, false, catch, function, return, null, catch, switch, var, let, const, if, in, while, do, else, case, break},
  keywords=[2]{class, export, boolean, throw, implements, import, this, await, async},
  keywords=[3]{setupTest, require, describe, beforeEach, resetAllStubs, basicClassTests, testCliUsage, testCliCalls},
  keywords=[4]{sts, RadvdConfigCli, testBasename},
  keywordstyle=\color{blue}\bfseries,
  keywordstyle=[2]\color{DarkOrchid}\bfseries,
  keywordstyle=[3]\color{yellow}\bfseries,
  keywordstyle=[4]\color{RoyalBlue}\bfseries,
  identifierstyle=\color{black},
  sensitive=false,
  comment=[l]{//},
  morecomment=[s]{/*}{*/},
  commentstyle=\color{ForestGreen}\ttfamily,
  stringstyle=\color{RubineRed}\ttfamily,
  morestring=[b]',
  morestring=[b]"
}
    % used by listings -> siehe header.text


\colorlet{punct}{red!60!black}
\definecolor{background}{HTML}{EEEEEE}
\definecolor{delim}{RGB}{20,105,176}
\colorlet{numb}{magenta!60!black}

\lstdefinelanguage{Json}{
    basicstyle=\normalfont\ttfamily,
    showstringspaces=false,
    breaklines=true,
    backgroundcolor=\color{background},
    literate=
      {":}{{{ "\color{punct}{:}}}}{1}
      {,}{{{\color{punct}{,}}}}{1}
      {\{}{{{\color{delim}{\{}}}}{1}
      {\}}{{{\color{delim}{\}}}}}{1}
      {[}{{{\color{delim}{[}}}}{1}
      {]}{{{\color{delim}{]}}}}{1}
      {Ö}{{\"O}}1
      {Ä}{{\"A}}1
      {Ü}{{\"U}}1
      {ß}{{\ss}}1
      {ü}{{\"u}}1
      {ä}{{\"a}}1
      {ö}{{\"o}}1
}
    
    \lstset{
       language={},
       backgroundcolor=\color{backgoundColor},
       extendedchars=true,
       basicstyle=\footnotesize\ttfamily,
       showstringspaces=false,
       showspaces=false,
       stepnumber=1,
       numbers=left,
       numberstyle=\footnotesize,
       numbersep=9pt,
       tabsize=2,
       breaklines=true,
       showtabs=false,
       captionpos=b,
       frame=single,
       rulecolor=\color{black},
       literate={✔}{$\checkmark$}{1}
              {Ö}{{\"O}}1
              {Ä}{{\"A}}1
              {Ü}{{\"U}}1
              {ß}{{\ss}}1
              {ü}{{\"u}}1
              {ä}{{\"a}}1
              {ö}{{\"o}}1
    }
    
    % Bezeichnung des Codes in der Unterschrift
        %\renewcommand{\lstlistingname}{Quellcode}
    % Überschrift des Quellcodeverzeichnis
        %\renewcommand{\lstlistlistingname}{Quellcodeverzeichnis}


%Anhängen von Pdf Dateien (muss nach xcolor geladen werden)
\usepackage{pdfpages}

% Drehen von Elemeneten
\usepackage{rotating}




% Glossary
     \usepackage[nonumberlist,  % Option entfernt die Seitenzahl im Glossary
                nopostdot,      % Kein Punkt am Ende (nötig für die Acronyme)
                acronym         % Gloassary wird auch für die Acronyme genutzt 
                ]{glossaries} 
    \makeglossaries

% Funktionale Programmierung für die eigenen Commands
\usepackage{xifthen}


% Help Counter for Pagenumbering
\newcounter{lastRomaPage}