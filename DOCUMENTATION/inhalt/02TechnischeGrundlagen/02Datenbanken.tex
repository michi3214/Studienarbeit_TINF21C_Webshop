\section{Datenbanken}
Datenbanken dienen zum persistenten und zentralen Ablegen von Daten. Durch die Datenbank werden die Daten einheitlich abgelegt und Zugriffe erfolgen über klare Schnittstellen. \citePage{kaufmann_sql-_2023}{2 f.}

Das verbreitetste Konzept sind relationale Datenbanken. Die Datenablage erfolgt in Tabellen (Relationen). Im Gegensatz dazu stehen die sogenannten NoSQL-Datenbanken. Hier werden die Daten nicht in feste Tabellen abgelegt, sondern unterliegen freieren Strukturen. Beispiele sind dokumentenorientierte Datenbanken. \citePage{buhler_datenmanagement_2019}{52}


\subsection{Relationale Datenbanken}
    Relationale Datenbanken nutzen wie bereits erwähnt eine Tabellen Struktur. Die Tabellen sortieren die Daten nach einzelnen Thematiken.
    
    
    Alle Daten werden in feste Tabellen eingetragen, welche die logischen Abhängigkeiten der Daten widerspiegelt. \todo{siehe Datenmanagement und erarbeiten einer Grafik mit Tabellen}

\subsection{NoSQL Datenbanken}
