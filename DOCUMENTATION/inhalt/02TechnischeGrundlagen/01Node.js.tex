\section{Node.js}
    JavaScript ermöglichte ursprünglich die Implementierung von Anwendungslogik auf dem Client. Von dem Server wurden neben statischen Inhalte, wie HTML und CSS auch JavaScript übertragen. JavaScript ist eine Programmiersprache und ermöglicht das dynamische anpassen der Webseite. Beispielsweise können Filter oder Sortierungen ausgeführt werden. Diese werden nun lokal auf dem Client ausgeführt und benötigen keine neuen Informationen von dem Server. \citePage{FullstackEntwicklung.Ackermann.2023}{34}

    Durch Node.js wurde das Einsatzgebiet von JavaScript von dem Client auf den Server ausgedehnt. Node.js ist eine Laufzeitumgebung für JavaScript und ermöglicht auch auf dem Server die Programmiersprache Node.js zu nutzen. \citePage{FullstackEntwicklung.Ackermann.2023}{436f.}
    
    Der sogenannte \gls{NPM} ermöglicht das Teilen von Programmcode unter den Programmieren. Es wird ermöglicht, einzelne Pakete, sogenannte Module zu installieren und diese innerhalb des eigenen Programms aufzurufen. Außerdem kann nach bestimmten Modulen gesucht oder die installierten Module aktualisiert werden. Somit können beispielsweise existierende Frameworks oder Funktionalitäten wiederverwendet werden, ohne eine eigene Implementierung vorzunehmen.  \citePage{Node.js.Springer.2020}{535}
    
    

    \subsection{React}
    
    \subsection{Next.js}

    
\begin{comment}
    - JavaScript und TypeScript mit beleuchten
\end{comment}