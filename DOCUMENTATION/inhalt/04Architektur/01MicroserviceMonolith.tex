\section{Microservices oder monolithische Architektur} \label{sec:MicroservicesOrMonolith}
\begin{comment}
    --> Microservices
        - Unterteilung
        - Vorteile gegenüber einem Monolithen 
\end{comment}

Bei der Architektur von Webanwendungen findet man meist einen Monolithische- oder eine Microsrvice-Architektur vor. Bei einer Monolithischen-Architektur wird die gesamte Anwendung als eine Einheit programmiert und bildet ein einziges Paket. Im Gegensatz dazu steht die Microservice-Architektur, welche die Anwendung in einzelne Services unterteilt. Beide Ansätze bieten unterschiedliche Vor- und Nachteile. \citePage{Node.js.Springer.2020}{669ff.} 



