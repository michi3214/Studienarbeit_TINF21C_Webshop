\section{Use Cases}

% Counter
    \newcounter{UseCaseCounter}
  
   % \DefineUserStory{Title}{Beschreibung}{Bild}{Label}
   %Labelpräfix UC-<refnr.> 
    \newcommand{\DefineUserStory}[4]{
        \refstepcounter{UseCaseCounter}
        \noindent\textbf{UC-\theUseCaseCounter ~#1}\storelabel{UC-\theUseCaseCounter ~#1}\label{Anf:#4} \\
        \noindent\rule[5mm]{\textwidth}{1pt} \\
        \vspace{-15mm} 
        
        \begin{center}
            \includegraphics[width=\textwidth]{#3}
            \captionof{figure}{UC-\theUseCaseCounter ~#1}
        \end{center}
        \noindent {#2}\nolinebreak
        \vspace{5mm}
    }

    \DefineUserStory{Account anlegen}{Für die Erstellung eiens Accounts benötigt der User bereits eine E-Mail Adresse, Facebook oder Google Konto. Zudem muss die entsprechende Seite auf der Webseite geöffnet worden sein.}{images/01Startseite.png}{createAccount}
    
    \DefineUserStory{Benutzerinformationen verwalten}{Um die Benutzerinformationen zu bearbeiten und Anzupassen muss sich der Nutzer zuerst anmelden.}{images/01Startseite.png}{changeProfile}
    
    \DefineUserStory{Anmelden}{Um sich anmelden zu können, muss vorher ein Benutzerkonto angelegt werden.}{images/01Startseite.png}{loginUser}
    
    \DefineUserStory{Produkte in den Warenkorb legen}{Nutzer können Produkte in den Warenkorb legen. Hierzu sind keine vorherigen Schritte nötig.}{images/01Startseite.png}{manageBasket}
    
    \DefineUserStory{Kaufvorgang}{Bevor ein Kauf eingeleiet werden kann, müssen sich Produkte im Warenkorb befinden.}{images/01Startseite.png}{buy}
    
    \DefineUserStory{Produkte erstellen}{Produkte können nur von Administratoren des Webshops hinzugefügt werden. Dazu wird als erstes ein Login benötigt.}{images/01Startseite.png}{addProduct}
    
    \DefineUserStory{Produkte entfernen}{Um Produkte zu entfernen muss sich der Administrator mit einem Account anmelden, der die entsprechenden BErechtigungen hat.}{images/01Startseite.png}{deleteProduct}
    
