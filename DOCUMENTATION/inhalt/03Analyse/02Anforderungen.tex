\section{Anforderungsanalyse}\label{sec:Anforderungen}
    Zur Definition der Prioritäten der Anforderungen wurden die Schlüsselwörter aus dem \gls{RFC} 2119 genutzt \parencite{Keywords.Bradner.1997}.
    Die englischen Schlüsselwörter (mit ihrer deutschen Übersetzung in den Klammern) sind wie folgt nach Priorität sortiert:
    \enquote{MUST} (MUSS), \enquote{MUST NOT} (DARF NICHT), \enquote{REQUIRED} (ERFORDERLICH), \enquote{SHALL} (SOLL), \enquote{SHALL NOT} (SOLL NICHT), \enquote{SHOULD} (SOLLTE), \enquote{SHOULD NOT} (SOLLTE NICHT), \enquote{RECOMMENDED} (EMPFOHLEN), \enquote{MAY} (DARF), \enquote{OPTIONAL} (OPTIONAL) \parencite{Keywords.Bradner.1997}.

    Aus den vorherigen Use Cases ergeben sich die folgenden Anforderungen.
    %\listofDefineNotFktAnforderung
    %\newpage
    %\listofDefineNotFktAnforderung
    % Dies ist eine Unterseite von 04 Anforderungen
% Die Reiehnfolge der Anforderungen darf nicht mehr angepasst werden
\subsubsection{\normalsize Funktionale Anforderungen}

    \DefineFktAnforderung{Weboberfläche}{Die Anwendung \underline{muss} über ein Frontend verfügen. Dieses soll als Webseite realisiert werden und den Benutzern alle notwendigen Informationen und Funktionen der Anwendung zur Verfügung stellen.}{Webpage}

    \DefineFktAnforderung{Verwaltung von Produkten}{Shop-Administratoren \underline{müssen} die Möglichkeit haben, über das Frontend Produkte hinzuzufügen, zu ändern oder zu löschen.}{ProductManagement}

    \DefineFktAnforderung{Registrierung von Nutzern}{Den Nutzern \underline{müssen} mindestens drei verschiedene Registrierungsverfahren zur Verfügung stehen. Dazu kann auch die Nutzung von Kontoinformationen Dritter gehören.}{RegisterUser}

    \DefineFktAnforderung{Kauf von Waren}{Registrierten Benutzern \underline{muss} der Kauf von Produkte möglich sein. Dabei ist die Verfügbarkeit der Produkte zu berücksichtigen.}{BuyProducts}

     \DefineFktAnforderung{Warenkorb}{Ein Nutzer \underline{muss} Waren in einen Warenkorb ablegen können, bevor diese gekauft werden. Zudem \underline{muss} es auch möglich sein, Produkte vor dem Kauf aus dem Warenkorb zu entfernen.}{ShoppingBasket}
    
    \DefineFktAnforderung{Verwaltung von eigenen Käufen}{Ein Benutzer \underline{sollte} seine eigenen Käufe einsehen können. Noch nicht bearbeitete und ältere Käufe sollten getrennt angezeigt werden.}{ManagementOwnBuys}

    \DefineFktAnforderung{Verwaltung von vorhandenen Käufen}{Shop-Administratoren \underline{sollten} in der Lage sein, Einkäufe, die noch nicht bearbeitet wurden, einzusehen und zu bearbeiten.}{ManagementExistingBuys}

    \DefineFktAnforderung{Nutzerverwaltung}{Optional \underline{sollten} Shop-Administratoren alle Nutzer einsehen und deren Nutzerdaten bearbeiten können.}{UserManagment}

\begin{comment}
    \RequirementInRef{Webpage}
    \RequirementInRef{ProductManagement}
    \RequirementInRef{RegisterUser}
    \RequirementInRef{BuyProducts}
    \RequirementInRef{ShoppingBasket}
    \RequirementInRef{ManagementOwnBuys}
    \RequirementInRef{ManagementExistingBuys}
    \RequirementInRef{UserManagment}
\end{comment}
    % Dies ist eine Unterseite von 04 Anforderungen
% Die Reiehnfolge der Anforderungen darf nicht mehr angepasst werden
\subsubsection{\normalsize Nicht-funktionale Anforderungen}

    \DefineNichtFktAnforderung{Sicherer Datenaustausch}{Die Anwendung sollte gewisse Sicherheitsfunktionen bieten. Es \underline{sollte} eine verschlüsselte Kommunikation durch TLS ermöglicht werden.}{SafetyDataTransfer}

    \DefineNichtFktAnforderung{Zusätzliche Authentifizierung der Nutzer}{Neben der Nutzung von einem Passwort ist auch die Nutzung von Zwei-Faktor-Authentisierungen für die Authentifizierung der Nutzer  empfehlenswert.}{UserAuthentification}

    \DefineNichtFktAnforderung{Benutzerfreundlichkeit}{Die Anwendung \underline{sollte} im Idealfall selbsterklärend sein und eine einfache Handhabung durch die Nutzer ermöglichen.}{Usability}

    \DefineNichtFktAnforderung{Portabilität}{Die Anwendung \underline{sollte} später auf andere Plattformen (Computer, mobile Endgeräte) übertragen werden können. Die Schaffung eines weiteren Zugangs auf die Anwendung über beispielsweise einem Mobiltelefon ist allerdings nicht Teil dieser Umsetzung.}{Portability}


\begin{comment}
    \RequirementInRef{SafetyDataTransfer}
    \RequirementInRef{UserAuthentification}
    \RequirementInRef{Usability}
    \RequirementInRef{Portability}
\end{comment}




\begin{comment}
    So kann man auf Anforderungen refernzieren
    \RequirementInRef{FJ-F-1}{Zuverlässige Zustellung der Nachrichten}
    \RequirementInRef{FJ-F-2}{Erkennung eines Knotenausfalls}
    \RequirementInRef{FJ-F-3}{Geordnete Zustellung der Ereignisse}
    \RequirementInRef{FJ-F-4}{Unterschiedliche Ereignistypen}
    \RequirementInRef{FJ-F-5}{Schnittstelle zum SEM Cache}
    \RequirementInRef{FJ-F-6}{Erkennung des Ereignisses durch M2000}
    \RequirementInRef{FJ-F-7}{Hochfahren einer Unit}

    \RequirementInRef{FJ-NF-1}{Ausfallsicherheit}
    \RequirementInRef{FJ-NF-2}{Anpassbarkeit und Wartbarkeit}
    \RequirementInRef{FJ-NF-3}{Support}
    \RequirementInRef{FJ-NF-4}{Lizenzen}
    \RequirementInRef{FJ-NF-5}{Implementierungsaufwand}
    \RequirementInRef{FJ-NF-6}{Koexistenz beider Implementierungen}
    \RequirementInRef{FJ-NF-7}{Betrieb im Mischcluster}
    \RequirementInRef{FJ-NF-8}{Datensicherheit}
    \RequirementInRef{FJ-NF-9}{Datenintegrität}
    \RequirementInRef{FJ-NF-10}{Universelle Schnittstellen für Producer}
    \RequirementInRef{FJ-NF-11}{Skalierbarkeit}
    \RequirementInRef{FJ-NF-12}{Updatemanager}
\end{comment}