% Dies ist die Hauptdatei. Von hier aus werden alle anderen Dateien und Daten geladen
\documentclass[
    ngerman, %Sprache
    hidelinks, % nicht nötig ? 
    chapteratlists=0pt % chapteratlists gibt den Abstand bei Verzeichnissen von Kapiteln zu Kapiteln an
    ]{scrreprt}
\KOMAoptions{
    fontsize = 12pt,
    titlepage = true,
    DIV=13
}

% In der header.text Datei erfolgen die Einbindungen aller Packages und damit verbundene Konfigurationen. 
%Die variable.tex beinhaltet Variablen der Ausarbeitung, um eine Anpassung an eine andere Arbeit zu erleichtern. 
%In der commands.tex Datei sind eigene Befehle hinterlegt. 
\usepackage[utf8]{inputenc}
\usepackage[T1]{fontenc}
\usepackage[ngerman]{babel}

% Ermöglicht das Erstellen von comment Blocks
    \usepackage{comment} % Kommentarblöcke mit \begin{comment}

% Eine Tabelle kann die größe einer Seite überschreiten
    \usepackage{longtable} % Used for a table longer one page

    
%Import von Bildern
    \usepackage{graphicx}
    % Position der Bilder
        \graphicspath{ {./images/} }
    % Ausrichten von Bildern 
        \usepackage[export]{adjustbox}


% unterschiedliche Schriftarten
    %\usepackage{lmodern}
    %\usepackage{palatino} 
    %\usepackage{tgheros}
    %\usepackage{goudysans}
    %\usepackage{libertine}
    %\usepackage[sfdefault]{AlegreyaSans}
        %\renewcommand*\oldstylenums[1]{{\AlegreyaSansOsF #1}} % gehört zu AlegreyaSans
    %\usepackage[sfdefault]{biolinum}
    %\usepackage{courier}


% Striche bei der Fuß-, sowie Kopfzeile
    \usepackage[footsepline,plainfootsepline, headsepline, plainheadsepline]{scrlayer-scrpage} 
    \pagestyle{scrheadings}
    \clearpairofpagestyles
    \KOMAoptions{
        headheight = 1cm,
        footheight = 1cm,
        singlespacing  =true
    }
    
% Größen der Überschriften
    \setkomafont{chapter}{\Large}
    \setkomafont{section}{\large}
    \setkomafont{subsection}{\normalsize}
    \setkomafont{subsubsection}{\small}

% viele Mathe-Symbole
    \usepackage{amssymb}
    % Erweiterungen für amsmath
        \usepackage{mathtools} 


% Links
    \usepackage{hyperref}
    
        % Durch diese Einstellung werden Links hervorgehoben durch die gegebenen Farben
        %\hypersetup{
        %   colorlinks,
        %   linkcolor={red!50!black}, % color of internal links
        %   citecolor={blue!50!black},% color of links to bibliography
        %   urlcolor={blue!80!black}  % color of external links
        %}
    
    % Referenz zu Bildern mit Ansicht auf Bilder
    \usepackage[figure]{hypcap}
    % Unter einer Abbildung steht nun Abb.
    \addto\captionsngerman{%
      \renewcommand{\figurename}{Abb.}%
    }

    % Referenzen in der Datei
        \usepackage[nameinlink]{cleveref}
                
    	% nicht nötig da ngerman in document class definiert
            \crefname{lstlisting}{Listing}{Listings}
            %\crefname{chapter}{Kapitel}{Kapitel}
            %\crefname{section}{Abschnitt}{Abschnitte}
            %\crefname{subsection}{Unterabschnitt}{Unterabschnitte}
            %\crefname{figure}{Abbildung}{Abbildungen}
        

% Ein Durchlauf bei der Erstellung von Verzeichnissen
\usepackage{bookmark}
\usepackage{makeidx}

% Einstellung des allgemeinen Zeilenabstands und die möglichkeit zur Anpassnung mit spacing
\usepackage[onehalfspacing]{setspace}

% Anführungszeichen
\usepackage[babel,german=quotes]{csquotes}

%Unterstrich für die Unterschrift / Tabellen erstellung
\usepackage{tabularx}
% Tabellenformatierung
    \usepackage{multirow}

% Literaturverweise
    \usepackage[style=authoryear, % Style Definition
                backend=biber,    % Backend Definition
                minnames=3,       % Minimale Anzahl an Namen
                maxbibnames=3,    % Maximale Anzahl an Namen
                dashed=false,     % 
                sorting=nty       % Sortierung des Verzeichnisses nach name, title, year.
                ]{biblatex}
    % In dieser Datei erfolgt die Definition der Quellen 
        % In dieser Dateien sind die Dateien mit den Literturinhalten angeben
        \addbibresource{verzeichnisse/Literatur/books.bib} 
        %\addbibresource{verzeichnisse/Literatur/internals.bib} 
        %\addbibresource{verzeichnisse/Literatur/website.bib} 
        %\addbibresource{verzeichnisse/Literatur/misc.bib} 
        %\addbibresource{verzeichnisse/Literatur/manual.bib} 
        \addbibresource{verzeichnisse/Literatur/RFC.bib} 
    % Komma im Text bei cite zwischen den Daten (Standard .)
        \DeclareDelimFormat{nameyeardelim}{\addcomma\space}
    % Komma im Literaturverzeichnis zwischen den Daten (Standard .)
        \renewcommand*{\newunitpunct}{\addcomma\space} 
    % Zeichen hinter jeder Literatur im Verzeichnis (Standard .)
        \renewcommand*{\finentrypunct }{} 
    % Bezeichnung bei mehreren Autoren
        \DefineBibliographyStrings{ngerman}{andothers = {{u. a.},}}
    %Abstand im Literaturverzeichnis
        % Allgemein
            \setlength{\bibitemsep}{0.2\baselineskip}
        % Gruppiert nach Namen
            \setlength{\bibnamesep}{0.6\baselineskip}
        % Abstand zwischen anderen Anfangsbuchstaben
            %\setlength{\bibinitsep}{0.6\baselineskip}



%Tabellen Einstellungen
    % Fußnote in einer Tabelle
    \usepackage{threeparttable}
    %Tabellen/Bilder Überschrift/Unterschrift
    \usepackage{caption}
    %Anpassen der Float-Umgebungen 
    \usepackage{float} % float kann mithilfe von capition entfernt werden (Nutze beides)
    %Umbennen der Bezeichner
    \addto\captionsngerman{
        \renewcommand{\tablename}{\small{Tab.}}
    }
    

%Farb Packet (Wird hauptsächlich beim Code highlighting genutzt) 
    \usepackage[dvipsnames]{xcolor}

% Für Codehighlighting
    \usepackage{listings}
    \definecolor{backgoundColor}{rgb}{.9,.9,.9} % lightgray 

    % Programmiersprachen Definitionen
    % used by listings -> siehe header.text

\definecolor{yellow}{rgb}{0.93, 0.57, 0.13}

\lstdefinelanguage{JavaScript}{
  %keywordstyle={[2]\color{White}},
  morekeywords={typeof, new, true, false, catch, function, return, null, catch, switch, var, let, const, if, in, while, do, else, case, break},
  keywords=[2]{class, export, boolean, throw, implements, import, this, await, async},
  keywords=[3]{setupTest, require, describe, beforeEach, resetAllStubs, basicClassTests, testCliUsage, testCliCalls},
  keywords=[4]{sts, RadvdConfigCli, testBasename},
  keywordstyle=\color{blue}\bfseries,
  keywordstyle=[2]\color{DarkOrchid}\bfseries,
  keywordstyle=[3]\color{yellow}\bfseries,
  keywordstyle=[4]\color{RoyalBlue}\bfseries,
  identifierstyle=\color{black},
  sensitive=false,
  comment=[l]{//},
  morecomment=[s]{/*}{*/},
  commentstyle=\color{ForestGreen}\ttfamily,
  stringstyle=\color{RubineRed}\ttfamily,
  morestring=[b]',
  morestring=[b]"
}
    % used by listings -> siehe header.text


\colorlet{punct}{red!60!black}
\definecolor{background}{HTML}{EEEEEE}
\definecolor{delim}{RGB}{20,105,176}
\colorlet{numb}{magenta!60!black}

\lstdefinelanguage{Json}{
    basicstyle=\normalfont\ttfamily,
    showstringspaces=false,
    breaklines=true,
    backgroundcolor=\color{background},
    literate=
      {":}{{{ "\color{punct}{:}}}}{1}
      {,}{{{\color{punct}{,}}}}{1}
      {\{}{{{\color{delim}{\{}}}}{1}
      {\}}{{{\color{delim}{\}}}}}{1}
      {[}{{{\color{delim}{[}}}}{1}
      {]}{{{\color{delim}{]}}}}{1}
      {Ö}{{\"O}}1
      {Ä}{{\"A}}1
      {Ü}{{\"U}}1
      {ß}{{\ss}}1
      {ü}{{\"u}}1
      {ä}{{\"a}}1
      {ö}{{\"o}}1
}
    
    \lstset{
       language={},
       backgroundcolor=\color{backgoundColor},
       extendedchars=true,
       basicstyle=\footnotesize\ttfamily,
       showstringspaces=false,
       showspaces=false,
       stepnumber=1,
       numbers=left,
       numberstyle=\footnotesize,
       numbersep=9pt,
       tabsize=2,
       breaklines=true,
       showtabs=false,
       captionpos=b,
       frame=single,
       rulecolor=\color{black},
       literate={✔}{$\checkmark$}{1}
              {Ö}{{\"O}}1
              {Ä}{{\"A}}1
              {Ü}{{\"U}}1
              {ß}{{\ss}}1
              {ü}{{\"u}}1
              {ä}{{\"a}}1
              {ö}{{\"o}}1
    }
    
    % Bezeichnung des Codes in der Unterschrift
        %\renewcommand{\lstlistingname}{Quellcode}
    % Überschrift des Quellcodeverzeichnis
        %\renewcommand{\lstlistlistingname}{Quellcodeverzeichnis}


%Anhängen von Pdf Dateien (muss nach xcolor geladen werden)
\usepackage{pdfpages}

% Drehen von Elemeneten
\usepackage{rotating}




% Glossary
     \usepackage[nonumberlist,  % Option entfernt die Seitenzahl im Glossary
                nopostdot,      % Kein Punkt am Ende (nötig für die Acronyme)
                acronym         % Gloassary wird auch für die Acronyme genutzt 
                ]{glossaries} 
    \makeglossaries

% Funktionale Programmierung für die eigenen Commands
\usepackage{xifthen}


% Help Counter for Pagenumbering
\newcounter{lastRomaPage}

% Informations about autor
    \author{Michael Grote} % can not used as normal variable 
    \newcommand{\autor}{Michael Grote}
    \newcommand{\martrikelnr}{6417568}

% Information about the school and company
    \newcommand{\course}{TINF21-C}
    %\newcommand{\company}{Fujitsu Services GmbH}
    %\newcommand{\locationCompany}{Munich}
    \newcommand{\school}{Dualen Hochschule Baden-Württemberg Stuttgart}

% Informations of the project
    \title{Webshop mit unterschiedlichen Registrierungsmöglichkeiten für die Nutzer}
    \newcommand{\projectTitle}{Webshop mit unterschiedlichen Registrierungsmöglichkeiten für die Nutzer}
    \newcommand{\projectTypeName}{Studienarbeit T3101}
    \newcommand{\projectType}{Studienarbeit}
    \newcommand{\projectName}{T3101}
    \newcommand{\studiengang}{Informatik (B.Sc.)}
    \newcommand{\location}{Stuttgart}

    \newcommand{\startDate}{02.10.2023}
    \newcommand{\lastDate}{13.06.2024} % \today for the day or manual date "März, 2022"
    \newcommand{\lastDateFormat}{13.Juni 2024}

    \newcommand{\supervisorSchool}{Herr Rafael Rietz}


% Diese Datei umfasst selbst definierte oder geänderte Befehle
    % List of commands
    %   Legende
    %   L = Label
    %   K = Kapitel
    %   Sek = Sektion
    %   S = Seite
    %   B = Bezeichnung
    %
    %   Gruppe  Bezeichnung Ergebnis
    %   Referenzen
    %       Kapitel
    %           \Uref{L}                | (siehe Kapitel 3.4)
    %           \Chref{L}               | (siehe Kapitel 3.4 Name)
    %           \ChInref{L}             | Kapitel 3.4 Name
    %       Anforderungen
    %           \RequirementInRef{L}{B} | Anforderung FJ-F-<NUMMER> Name
    %           \RequirementRef{L}{B}   | (siehe Anforderung FJ-F-<NUMMER> Name)
    %       Bilder 
    %           \PicRef{L}              | Abb. <Nummer> <Kursiv Beschreibung>
    %           \PicInRef{L}            | Abb. <Nummer>
    %       Zitate                      
    %           \citeCh{L}{K}           | (..., siehe Kap. ...)
    %           \citeChSec{L}{K}{Sek}   | (..., siehe Kap. ... Abs. ....)
    %           \citePage{L}{S}         | (..., vgl. S. ...)
    %
    %
    %   Kapitel
    %      Erstellt einen Eintrag im Inhaltsverzeichnis und vor dem Kapitel
    %       \DefineNewChapter{Name} | Geeignet für Einleitung, Sperrvermerk, Anhang, Abkürzungsverzeichnis
    %       \DefineNewListOf[label]{Name}  | Geeignet für Abbildungsverzeichnis, Tabellenverzeichnis, ...
    %

%   Listen
%       \DefineFktAnforderung{Headline]{Beschreibung}{Labelsuffix}
%       \DefineNichtFktAnforderung{Headline}{Beschreibung}{Labelsuffix}

%\usepackage{tocloft}% http://ctan.org/pkg/tocloft

\makeatletter % needed because of the @
    \newcommand\storelabel[1]{\def\@currentlabel{#1}}
\makeatother



\newcounter{DefineFktAnforderung}
\setcounter{DefineFktAnforderung}{0}

% \AnforderungDesign{Präfix}{Headline}{Text}{Label}{Counter}
\newcommand{\AnforderungDesign}[5]{
    \noindent\textbf{#1-#5 ~#2}\storelabel{#1-#5 ~#2}\label{Anf:#4} \\
    \noindent\rule[5mm]{\textwidth}{1pt} \\
    \vspace{-15mm} 
        
    \noindent {#3}\nolinebreak
    \vspace{5mm}
}

%\newlistof{DefineFktAnforderung}{DefineFktAnforderungDef}{\small Liste der funktionalen Anforderungen}
% \DefineFktAnforderung{Headline}{Describtion}{Labelsuffix}
\newcommand{\DefineFktAnforderung}[3]{
    \refstepcounter{DefineFktAnforderung}

    \AnforderungDesign{F}{#1}{#2}{#3}{\theDefineFktAnforderung} \\
    
    %\addcontentsline{DefineFktAnforderungDef}{figure}
    %{\protect\numberline{F-\theDefineFktAnforderung}#1}\par%
}


\newcounter{DefineNotFktAnforderung}
\setcounter{DefineNotFktAnforderung}{0}

%\newlistof{DefineNotFktAnforderung}{DefineNotFktAnforderungDef}{\small Liste der nicht-funktionalen Anforderungen}
% \DefineFktAnforderung{Headline}{Describtion}{Labelsuffix}
\newcommand{\DefineNichtFktAnforderung}[3]{
    \refstepcounter{DefineNotFktAnforderung}

    \AnforderungDesign{NF}{#1}{#2}{#3}{\theDefineNotFktAnforderung} \\
    
    %\addcontentsline{DefineNotFktAnforderungDef}{figure}
    %{\protect\numberline{F-\theDefineNotFktAnforderung}#1}\par%
}


%Referenzen 
    % Universal Referenz mit angepasstem Text (siehe Kapitel 3.4)
        \newcommand{\Uref}[1]{(siehe \cref{#1})}
    % Kapitel Referenz mit angepasstem Text (siehe Kapitel 3.4 Name)
        \newcommand{\ChRef}[1]{(siehe \cref{#1} \textit{\nameref{#1}})}
    % Kapitel Referenz Inline mit angepasstem Text Kapitel 3.4 Name
        \newcommand{\ChInRef}[1]{\cref{#1} \textit{\nameref{#1}}}

        
    % Bild Referenz mit angepasstem Text Abb. <Nummer> <Kursiv Beschreibung>
        \newcommand{\PicRef}[1]{\cref{#1} \textit{\nameref{#1}}}
    % Bild Referenz mit angepasstem Text Abb. <Nummer> 
        \newcommand{\PicInRef}[1]{\cref{#1}}

        
    % Anforderungs Referenz mit angepasstem Text (siehe Anforderung F-<NUMMER> Name)
        \newcommand{\RequirementRef}[1]{(siehe Anforderung \textit{\ref{Anf:#1}}}
    % Anforderungs Referenz mit angepasstem Text Anforderung F-<NUMMER> Name
        \newcommand{\RequirementInRef}[1]{Anforderung \textit{\ref{Anf:#1}}}

% Zitierung
    % Kapitel zietieren (..., siehe Kap. ...)
        \newcommand{\citeCh}[2]{\parencite[siehe Kap. #2]{#1}}
    % Kapitel mit Sektion zietieren (..., siehe Kap. ... Abs. ....)
        \newcommand{\citeChSec}[3]{\parencite[siehe Kap. #2 Abs. #3]{#1}}
    % Mit Seiten zietieren (..., vgl. S. ...)
        \newcommand{\citePage}[2]{\parencite[vgl. S. #2]{#1}}





% Anpassung der Chapter im Inhaltsverzeichnis 
    % \DefineNewChapter{Name}
    % (geeignet für Einleitung, Sperrvermerk, Anhang, ...)
    \newcommand{\DefineNewChapter}[1]{
        \phantomsection 
        \addtocontents{toc}{\vspace{-1.5ex}}
        \addcontentsline{toc}{chapter}{#1}
        \addchap*{#1}
    }

    % (geeignet für Abbildungsverzeichnis, Tabellenverzeichnis, ...)
    % \DefineNewListOf{label}[Name]
    \newcommand{\DefineNewListOf}[2][]{
        \phantomsection 
        \ifthenelse{\isempty{#1}}{}{\label{#1}}
        \addtocontents{toc}{\vspace{-1.5ex}}
        \addcontentsline{toc}{chapter}{#2}
    }





% Debugging und Nützlich 
    %\usepackage{showframe} % Rahmen um Blockelemente
    %\usepackage{labelschanged} % Fehler mit Labeln werden besser geloggt
    %\usepackage{showkeys} % Label werden in der Pdf angezeigt 
    \usepackage[german,textsize=tiny, colorinlistoftodos, disable]{todonotes} % Zum einfügen von TODOs hilfreich
    % Parameter disable um es zu verstecken
        \setlength{\marginparwidth}{2cm} % Legt die breite für die Seitennotizen fest



\begin{document}
    %TODO List (muss später auskommentiert werden)
    \listoftodos 
    
    %Deckblatt
    \include{inhalt/00Deckblatt.tex}

    %Abschalten der Linien von der Kopf- und Fußzeile
    \KOMAoptions{headsepline=false, plainheadsepline=false, footsepline=true, plainfootsepline=true}

    %Abstand des Chapters zum Seitenanfang
    \renewcommand*\chapterheadstartvskip{\vspace*{-\topskip}}
    \renewcommand*\chapterheadendvskip{\vspace*{1\baselineskip plus .1\baselineskip minus .167\baselineskip}}

    % Nummerierung der Verzeichnis
    \pagenumbering{Roman}
    \cfoot*{\pagemark}
    
    %Erklärung
    \addchap{Erklärung}
% https://www.dhbw-stuttgart.de/studierendenportale/informatik/Dokumente/Erklaerung.pdf

Ich versichere hiermit, dass ich die vorliegende Arbeit selbstständig verfasst und keine anderen als die angegebenen Quellen und Hilfsmittel benutzt habe.

Falls sowohl eine gedruckte als auch eine elektronische Fassung abgegeben wurde, 
versichere ich zudem, dass die eingereichte elektronische Fassung mit der gedruckten Fassung übereinstimmt.\bigskip \\

\parbox{6cm}{\centering \location, \today \hrule
\strut \centering\footnotesize Ort, Datum} \hfill\parbox{6cm}{\hrule
\strut \centering\footnotesize Unterschrift}


    %Kurzzusammenfassung
    
% Beeinflusst den Abstand im Inhaltsverzeichnis
\DefineNewChapter{Zusammenfassung}

    \subsection*{Deutsch} 
      

    \subsection*{Englisch}
   



    %Abstract
    
% Beeinflusst den Abstand im Inhaltsverzeichnis
%\DefineNewChapter{Abstract}

   % Inhaltsverzeichnis
    \begin{spacing}{1}
        %\addcontentsline{toc}{chapter}{Inhaltsverzeichnis}
      \tableofcontents
    \end{spacing}
    \begin{spacing}{1.5}

    % Abkürzungsverzeichnis
    \cleardoublepage
    % Eintrag im Inhaltsverzeichnis
        \DefineNewListOf{Abkürzungsverzeichnis}
    %Technsiche Grundlagen
    % Node.js
        \newacronym{RFC}{RFC}{Request for Comments}
        \newacronym{NPM}{NPM}{Node Package Manager}
    

    \printglossary[title=Abkürzungsverzeichnis,style=index,nogroupskip, type=\acronymtype]
    %%Technsiche Grundlagen
    % Node.js
        \newacronym{RFC}{RFC}{Request for Comments}
        \newacronym{NPM}{NPM}{Node Package Manager}
    

    
    %Abbildungsverzeichnis
    \cleardoublepage
    % Eintrag im Inhaltsverzeichnis
        \DefineNewListOf{Abbildungsverzeichnis}
    \listoffigures
    
    % Tabellenverzeichnis
   % \cleardoublepage
     % Eintrag im Inhaltsverzeichnis
        \DefineNewListOf{Tabellenverzeichnis}
   \listoftables

    % Quellcodeverzeichnis
   % \cleardoublepage
    % Eintrag im Inhaltsverzeichnis
        \DefineNewListOf{Listings} %% German: Quellcodeverzeichnis
    \lstlistoflistings

    %Glossar
    \cleardoublepage
    % Eintrag im Inhaltsverzeichnis
        \DefineNewListOf{Glossar} 
    \printglossary[style=altlist]
    % SE Infrastrucktur
    \newglossaryentry{M2000}
    {
        name=M2000,
        plural=M2000,
        description={Das \gls{M2000} ist die Basis Software auf den Management \glspl{Unit} der SE Infrastruktur \parencite{ManualSEInfrastrukturStrukturderSoftware.FujitsuTechnologySolutionsGmbH.2023}.
        }
    }

    \newglossaryentry{Client}
    {
        name=Client,
        plural=Clients,
        description={}
    }

    \newglossaryentry{Server}
    {
        name=Server,
        plural=Server,
        description={}
    }
    
    % Gender Hinweis
    %\include{inhalt/00Verzeichnisse/05Gender}
    
    \end{spacing}
    
    
    %Einschalten der Linien von der Kopf- und Fußzeile
    \cleardoublepage
    \KOMAoptions{headsepline=true, plainheadsepline=true, footsepline=true, plainfootsepline=true}
        
    %Inhalt
    \begin{spacing}{1.5}
    
        %Seitennummerierung
        \cleardoublepage
        \setcounter{lastRomaPage}{\value{page}}
        \pagenumbering{arabic}
        \setcounter{page}{1}
        \cfoot*{Seite \thepage\ von \pageref{LastPage} }
    
        % Chapter am oberen rechten Rand
        \automark{chapter}
        \ohead*{\headmark}
        
        % Einleitung
        \chapter{Einleitung}

    \section{Zielsetzung der Arbeit}
     Durch die 
    
    \section{Zielsetzung der Software}
    Dies ist wohl die Zielsetzung deiner Arbeit. 

     
     

        % Technische Grundlagen
        \chapter{Technische Grundlagen}

\section{Node.js}
    JavaScript ermöglichte ursprünglich die Implementierung von Anwendungslogik auf dem Client. Von dem Server wurden neben statischen Inhalte, wie HTML und CSS auch JavaScript übertragen. JavaScript ist eine Programmiersprache und ermöglicht das dynamische anpassen der Webseite. Beispielsweise können Filter oder Sortierungen ausgeführt werden. Diese werden nun lokal auf dem Client ausgeführt und benötigen keine neuen Informationen von dem Server. \citePage{FullstackEntwicklung.Ackermann.2023}{34}

    Durch Node.js wurde das Einsatzgebiet von JavaScript von dem Client auf den Server ausgedehnt. Node.js ist eine Laufzeitumgebung für JavaScript und ermöglicht auch auf dem Server die Programmiersprache Node.js zu nutzen. \citePage{FullstackEntwicklung.Ackermann.2023}{436f.}
    
    Der sogenannte \gls{NPM} ermöglicht das Teilen von Programmcode unter den Programmieren. Es wird ermöglicht, einzelne Pakete, sogenannte Module zu installieren und diese innerhalb des eigenen Programms aufzurufen. Außerdem kann nach bestimmten Modulen gesucht oder die installierten Module aktualisiert werden. Somit können beispielsweise existierende Frameworks oder Funktionalitäten wiederverwendet werden, ohne eine eigene Implementierung vorzunehmen.  \citePage{Node.js.Springer.2020}{535}
    
    

    \subsection{React}
    
    \subsection{Next.js}

    
\begin{comment}
    - JavaScript und TypeScript mit beleuchten
\end{comment}
\section{Datenbanken}

\subsection{Relationale Datenbanken}

\subsection{No SQL Datenbanken}




        
        % Analyse
        \chapter{Analyse}

\section{Use Cases}

% Counter
    \newcounter{UseCaseCounter}
  
   % \DefineUserStory{Title}{Beschreibung}{Bild}{Label}
   %Labelpräfix UC-<refnr.> 
    \newcommand{\DefineUserStory}[4]{
        \refstepcounter{UseCaseCounter}
        \noindent\textbf{UC-\theUseCaseCounter ~#1}\storelabel{UC-\theUseCaseCounter ~#1}\label{Anf:#4} \\
        \noindent\rule[5mm]{\textwidth}{1pt} \\
        \vspace{-15mm} 
        
        \begin{center}
            \includegraphics[width=0.6\textwidth]{#3}
            \captionof{figure}{UC-\theUseCaseCounter ~#1}
        \end{center}
        \noindent {#2}\nolinebreak
        \vspace{5mm}
    }

    \DefineUserStory{Account anlegen}{Für die Erstellung eiens Accounts benötigt der User bereits eine E-Mail Adresse, Facebook oder Google Konto. Je nach genutzten Verfahren müssen noch weitere benötigte Informationen hinterlegt werden.}{images/UseCases/UseCase1AccountAnlegen.png}{createAccount}
    
    \DefineUserStory{Anmelden}{Um sich als Nutzer anzumelden, muss zuvor der Registierungsprozess abgeschlossen sein. Nach einer erfolgreichen Registrierung kann sich der Nutzer mit den Daten auf der Webseite anmelden.}{images/UseCases/UseCase2Login.png}{loginUser}
    
    \DefineUserStory{Benutzerinformationen verwalten}{Persönliche Informationen und Daten können durch den Nutzer verändert werden. Beispielsweise kann die Adresse des Nutzers angepasst werden. Die Änderungen erfordert eine vorherige erfolgreiche Anmeldung.}{images/UseCases/UseCase3BenutzerinformationenVerwalten.png}{changeProfile}
    
    \DefineUserStory{Produkte in den Warenkorb legen}{Nutzer können Waren in den Warenkorb legen. Hierzu muss der Nutzer nicht angemeldet oder registriert sein.}{images/UseCases/UseCase4WarenkorbNutzen.png}{manageBasket}
    
    \DefineUserStory{Kaufvorgang}{Der Kaufvorrgang erfordert mindestens ein Produkt im Warenkorb und kann auf zwei Wegen geschehen. Der Nutzer kann ohne Registierung und Anmeldung einen Kauf vornehmen. Dazu steht der sogenannte Gastaccount zur Verfügung. Der zweite Weg ist die Registrierung oder Anmeldung an einem Nutzerkonto. Nach diesen Schritten kann der Kauf abgeschlossen werden. Der Kauf wird danach an die Logistik übermittelt für den Versand.}{images/UseCases/UseCase5Kaufen.png}{buy}
    
    \DefineUserStory{Produkte erstellen}{\todo{Abbildung anpassen: vorheriger Login}
    Neue Produkte können nur von Administratoren des Webshops hinzugefügt werden. Hierzu muss sich der Adminsitrator vorher einloggen}{images/UseCases/UseCase6Produkterstellen.png}{addProduct}
   


    \DefineUserStory{Produkte entfernen}{Auch das entfernen von Produkten, welche noch nicht ausverkauft sind, kann nur durch einen berechtigten Administrator erfolgen. Vor dem Entfernen eines Produktes muss sich der Administrator daher anmelden.}{images/UseCases/UseCase7Produktentfernen.png}{deleteProduct}
    


\section{Anforderungsanalyse}\label{sec:Anforderungen}
    Zur Definition der Prioritäten der Anforderungen wurden die Schlüsselwörter aus dem \gls{RFC} 2119 genutzt \parencite{Keywords.Bradner.1997}.
    Die englischen Schlüsselwörter (mit ihrer deutschen Übersetzung in den Klammern) sind wie folgt nach Priorität sortiert:
    \enquote{MUST} (MUSS), \enquote{MUST NOT} (DARF NICHT), \enquote{REQUIRED} (ERFORDERLICH), \enquote{SHALL} (SOLL), \enquote{SHALL NOT} (SOLL NICHT), \enquote{SHOULD} (SOLLTE), \enquote{SHOULD NOT} (SOLLTE NICHT), \enquote{RECOMMENDED} (EMPFOHLEN), \enquote{MAY} (DARF), \enquote{OPTIONAL} (OPTIONAL) \parencite{Keywords.Bradner.1997}.

    Aus den vorherigen Use Cases ergeben sich die folgenden Anforderungen.
    %\listofDefineNotFktAnforderung
    %\newpage
    %\listofDefineNotFktAnforderung
    % Dies ist eine Unterseite von 04 Anforderungen
% Die Reiehnfolge der Anforderungen darf nicht mehr angepasst werden
\subsubsection{\normalsize Funktionale Anforderungen}

    \DefineFktAnforderung{Weboberfläche}{Die Anwendung \underline{muss} über ein Frontend verfügen. Dieses soll als Webseite realisiert werden und den Benutzern alle notwendigen Informationen und Funktionen der Anwendung zur Verfügung stellen.}{Webpage}

    \DefineFktAnforderung{Verwaltung von Produkten}{Shop-Administratoren \underline{müssen} die Möglichkeit haben, über das Frontend Produkte hinzuzufügen, zu ändern oder zu löschen.}{ProductManagement}

    \DefineFktAnforderung{Registrierung von Nutzern}{Den Nutzern \underline{müssen} mindestens drei verschiedene Registrierungsverfahren zur Verfügung stehen. Dazu kann auch die Nutzung von Kontoinformationen Dritter gehören.}{RegisterUser}

    \DefineFktAnforderung{Kauf von Waren}{Registrierten Benutzern \underline{muss} der Kauf von Produkte möglich sein. Dabei ist die Verfügbarkeit der Produkte zu berücksichtigen.}{BuyProducts}

     \DefineFktAnforderung{Warenkorb}{Ein Nutzer \underline{muss} Waren in einen Warenkorb ablegen können, bevor diese gekauft werden. Zudem \underline{muss} es auch möglich sein, Produkte vor dem Kauf aus dem Warenkorb zu entfernen.}{ShoppingBasket}
    
    \DefineFktAnforderung{Verwaltung von eigenen Käufen}{Ein Benutzer \underline{sollte} seine eigenen Käufe einsehen können. Noch nicht bearbeitete und ältere Käufe sollten getrennt angezeigt werden.}{ManagementOwnBuys}

    \DefineFktAnforderung{Verwaltung von vorhandenen Käufen}{Shop-Administratoren \underline{sollten} in der Lage sein, Einkäufe, die noch nicht bearbeitet wurden, einzusehen und zu bearbeiten.}{ManagementExistingBuys}

    \DefineFktAnforderung{Nutzerverwaltung}{Optional \underline{sollten} Shop-Administratoren alle Nutzer einsehen und deren Nutzerdaten bearbeiten können.}{UserManagment}

\begin{comment}
    \RequirementInRef{Webpage}
    \RequirementInRef{ProductManagement}
    \RequirementInRef{RegisterUser}
    \RequirementInRef{BuyProducts}
    \RequirementInRef{ShoppingBasket}
    \RequirementInRef{ManagementOwnBuys}
    \RequirementInRef{ManagementExistingBuys}
    \RequirementInRef{UserManagment}
\end{comment}
    % Dies ist eine Unterseite von 04 Anforderungen
% Die Reiehnfolge der Anforderungen darf nicht mehr angepasst werden
\subsubsection{\normalsize Nicht-funktionale Anforderungen}

    \DefineNichtFktAnforderung{Sicherer Datenaustausch}{Die Anwendung sollte gewisse Sicherheitsfunktionen bieten. Es \underline{sollte} eine verschlüsselte Kommunikation durch TLS ermöglicht werden.}{SafetyDataTransfer}

    \DefineNichtFktAnforderung{Zusätzliche Authentifizierung der Nutzer}{Neben der Nutzung von einem Passwort ist auch die Nutzung von Zwei-Faktor-Authentisierungen für die Authentifizierung der Nutzer  empfehlenswert.}{UserAuthentification}

    \DefineNichtFktAnforderung{Benutzerfreundlichkeit}{Die Anwendung \underline{sollte} im Idealfall selbsterklärend sein und eine einfache Handhabung durch die Nutzer ermöglichen.}{Usability}

    \DefineNichtFktAnforderung{Portabilität}{Die Anwendung \underline{sollte} später auf andere Plattformen (Computer, mobile Endgeräte) übertragen werden können. Die Schaffung eines weiteren Zugangs auf die Anwendung über beispielsweise einem Mobiltelefon ist allerdings nicht Teil dieser Umsetzung.}{Portability}


\begin{comment}
    \RequirementInRef{SafetyDataTransfer}
    \RequirementInRef{UserAuthentification}
    \RequirementInRef{Usability}
    \RequirementInRef{Portability}
\end{comment}




\begin{comment}
    So kann man auf Anforderungen refernzieren
    \RequirementInRef{FJ-F-1}{Zuverlässige Zustellung der Nachrichten}
    \RequirementInRef{FJ-F-2}{Erkennung eines Knotenausfalls}
    \RequirementInRef{FJ-F-3}{Geordnete Zustellung der Ereignisse}
    \RequirementInRef{FJ-F-4}{Unterschiedliche Ereignistypen}
    \RequirementInRef{FJ-F-5}{Schnittstelle zum SEM Cache}
    \RequirementInRef{FJ-F-6}{Erkennung des Ereignisses durch M2000}
    \RequirementInRef{FJ-F-7}{Hochfahren einer Unit}

    \RequirementInRef{FJ-NF-1}{Ausfallsicherheit}
    \RequirementInRef{FJ-NF-2}{Anpassbarkeit und Wartbarkeit}
    \RequirementInRef{FJ-NF-3}{Support}
    \RequirementInRef{FJ-NF-4}{Lizenzen}
    \RequirementInRef{FJ-NF-5}{Implementierungsaufwand}
    \RequirementInRef{FJ-NF-6}{Koexistenz beider Implementierungen}
    \RequirementInRef{FJ-NF-7}{Betrieb im Mischcluster}
    \RequirementInRef{FJ-NF-8}{Datensicherheit}
    \RequirementInRef{FJ-NF-9}{Datenintegrität}
    \RequirementInRef{FJ-NF-10}{Universelle Schnittstellen für Producer}
    \RequirementInRef{FJ-NF-11}{Skalierbarkeit}
    \RequirementInRef{FJ-NF-12}{Updatemanager}
\end{comment}

        % Architektur 
        \chapter{Architektur} \label{chap:Architektur}

\section{Microservices oder monolithische Architektur} \label{sec:MicroservicesOrMonolith}
\begin{comment}
    --> Microservices
        - Unterteilung
        - Vorteile gegenüber einem Monolithen 
\end{comment}

Bei der Architektur von Webanwendungen findet man meist einen Monolithische- oder eine Microsrvice-Architektur vor. Bei einer Monolithischen-Architektur wird die gesamte Anwendung als eine Einheit programmiert und bildet ein einziges Paket. Im Gegensatz dazu steht die Microservice-Architektur, welche die Anwendung in einzelne Services unterteilt. Beide Ansätze bieten unterschiedliche Vor- und Nachteile. \citePage{Node.js.Springer.2020}{669ff.} 




\section{Datenbanken}

\subsection{Relationale Datenbanken}

\subsection{No SQL Datenbanken}
\section{JavaScript und TypeScript}

\begin{comment}
    - Unterschiede und EInsatz von JavaScript und TypeScript
\end{comment}

        % Konzeptionierung 
        \chapter{Konzeptionierung}\label{chap:Konzeptionierung}


\section{Microservices} \label{sec:Microservices}
\begin{comment}
    --> Genaue Bausteine der Implementierung
    --> einzelne Microservices näher betrachten
\end{comment}




\section{Sicherheit} \label{sec:Architektur}
\begin{comment}
    --> TLS
    --> User Anmeldung / Registrierung
        - Google / Facebook / Microsoft
\end{comment}

    \subsection{Registrierung und Anmeldung}
        \subsubsection{Nutzung von Kontoinformationen von Dritten}
        \subsubsection{Zwei-Faktor-Authentisierungen}

    \subsection{Verschlüsselung der Datenübermittlung}


\section{Aufbau der Datenbanken}


















        % Implementierung
        \chapter{Implementierung}

\begin{comment}
    CI/CD Pipline
\end{comment}

        % Reflexion
        \chapter{Reflexion und Ausblick}
\phantomsection

\label{LastPage} % definiert die letzte Seite, hiermit wird die Anzeige Page .. of ... realisiert
        

    \end{spacing}

    % Reset der Seitenzahlen für Verzeichnisse
    \cleardoublepage
    \pagenumbering{Roman}
    \setcounter{page}{\value{lastRomaPage}}

    %Abschalten der Linien von der Kopf- und Fußzeile
    \KOMAoptions{headsepline=false, plainheadsepline=false}
    
    % Literaturverzeichnis
    \cleardoublepage
        %Bib style
        \defbibheading{headLit}{\addchap{Literaturverzeichnis}}
        \printbibliography[heading=headLit]
    
    \newpage
   

    % Anhang (Bilder und Quellcode)
    \DefineNewChapter{Anhang} 

    
    % Bilder
    %

    % Quellcode
    %
     
    
    % Anhang für mögliche Dateien (need to add pdfpages as package in header file)
    %\includepdf[pages={1-3}]{anhang/}

  



   

    
\end{document}